\documentclass[a4paper,12pt]{article}

\usepackage[utf8]{inputenc}
\usepackage[T1]{fontenc}
\usepackage[greek,english]{babel}
\usepackage{geometry}
\usepackage{enumitem}
\usepackage{alphabeta}
\usepackage{lmodern}
\usepackage{textcomp}
\usepackage{amsmath}
\usepackage{amssymb}
\usepackage{amsfonts}
% \usepackage[hidelinks]{hyperref}
\usepackage{tabularx}
\usepackage[dvipsnames]{xcolor}
\usepackage{listings}
\usepackage{graphicx}
\usepackage{float}
\graphicspath{{figures/}} % Path to images


%%% Font of characters %%
%\usepackage{fontspec}
%\setmainfont{cmun}[
%  Extension=.otf,
%  UprightFont=*rm,
%  ItalicFont=*ti,
%  BoldFont=*bx,
%  BoldItalicFont=*bi,
%]
%\setsansfont{cmun}[
%  Extension=.otf,
%  UprightFont=*ss,
%  ItalicFont=*si,
%  BoldFont=*sx,
%  BoldItalicFont=*so,
%]
%\setmonofont{cmun}[
%  Extension=.otf,
%  UprightFont=*btl,% light version
%  ItalicFont=*bto,%  light version
%  BoldFont=*tb,
%  BoldItalicFont=*tx,
%]

%\geometry{
%a4paper,
%width=170mm,
%top=25mm,
%bottom=25mm
%}

% \input{avrListing.tex}

% \lstdefinestyle{ListingSample}{
% 	basicstyle=\small\ttfamily,
% 	numbers=none,
% 	keywordstyle=\color{blue}\bfseries,
% 	morekeywords={begin,end,for,maxint,to,do},
% 	% pos=l,
% }

\setlength\parindent{0pt} % Removes all indentation from paragraphs

\newcommand{\HRule}{\rule{\linewidth}{0.5mm}}

%----------------------------------------------------------------------------------------
% DOCUMENT INFORMATION
%----------------------------------------------------------------------------------------

\begin{document}

% \begin{titlepage}
\centering
\begin{figure}[h!]
\centering
\includegraphics[scale=1]{tuc}
\end{figure}
\textbf{\large ΠΟΛΥΤΕΧΝΕΙΟ ΚΡΗΤΗΣ}\\[0.3CM]
ΣΧΟΛΗ ΗΛΕΚΤΡΟΛΟΓΩΝ ΜΗΧΑΝΙΚΩΝ ΚΑΙ ΜΗΧΑΝΙΚΩΝ ΥΠΟΛΟΓΙΣΤΩΝ\\[1cm]
\HRule\\[0.15cm]
\Large{\textbf{ΕΝΣΩΜΑΤΩΜΕΝΑ ΣΥΣΤΗΜΑΤΑ ΜΙΚΡΟΕΠΕΞΕΡΓΑΣΤΩΝ}}\\[0.05cm]
\HRule\\[1cm]
\Large{\textbf{Εργασία:}}\\
\Large{Αναφορά Πρότζεκτ Sudoku}\\[1.5cm]
\begin{minipage}{0.9\textwidth}
\begin{flushleft}
\textbf{Φοιτητές:}\\
Βασιλείου Παναγιώτης$~~~~~~~$Ιωάννης-Ιάσων Γεωργακάς\\
2017030067$~~~~~~~~~~~~~~~~~~~~~~$2017030021\\[0.5cm]
% \large {Φοιτητής:}\\
% \large Βασιλείου Παναγιώτης\\
% AM: 2017030067\\
\end{flushleft}
\end{minipage}\\[2cm]
{\large\today}\\
\end{titlepage}

% \newpage
% \tableofcontents
% \newpage

\section{Εισαγωγή}
Σκοπός του Project είναι η επίλυση ενός παιχνιδιού Sudoku (9x9 πλέγμα) χρησιμοποιώντας έναν μικροελεγκτή AVR. Η διεπαφή του μικροελεγκτή με τον εξωτερικό κόσμο υλοποιείται με χρήση της σειριακής θύρας RS232 και με τη χρήση του τερματικού προγραμμάτος putty. Η επίλυση του Sudoku θα πραγματοποιηθεί σε μεταγενέστερο χρόνο χρησιμοποιώντας έναν άλγόριθμο οποίος και θα αποφασίστει με βάση την απόδοση του.

\section{Τεχνολογία}
Για την υλοποίηση του project χρησιμοποιήθηκε η πλακέτα STK500 με τον μικροελεγκτή ATmega16L με συχνότητα ρολογιού 10MHz (εξωτερικός κρύσταλλος). Τα μοντέλα του μικροελεκτή και του κρυστάλλου τηρούν προδιαγραφές οι οποίες είναι κοινές για όλες τις ομάδες. Η συγγραφή του κώδικα πραγματοποιήθηκε στη γλώσσα C με τον avr-gcc compiler, ενώ για την προσομοίωση του κώδικα χρησιμοποιήθηκε η πλατφόρμα Microchip Studio. Για την επικοινωνία της πλακέτα με τον εξωτερικό κόσμο χρησιμοποιήθηκε ένα καλώδιο USB σε Serial RS232 (βύσμα DB9).   


\section{Περιγραφή της υλοποίησης}
Το πρότζεκτ αποτελείται απο 4 μέρη τα οποία είναι η διεπαφή της σειριακής θύρας, η επεξεργασία των εντολών, ο αλγόριθμος επίλυσης του Sudoku και η οθόνη η οποία εμφανίζει τη πρόοδο επίλυσης του Sudoku. 

\subsection{Σειριακή Θύρα RS232}
Κατά την υλοποίηση της 

\subsection{Επεξεργασία των εντολών}

\subsection{Αλγόριθμός Επίλυσης Sudoku}

\subsection{Οθόνη - Progress Bar}
Για την απεικόνιση της προόδου επίλυσης του Sudoku χρησιμοποιήθηκαν τα LEDs κοινής ανόδου (CA) της πλακέτας. Συγκεκριμένα κατά τη διάρκεια επίλυσης του Sudoku πραγματοποιείται περιοδικά έλεγχος σχετικά με το πόσα κελιά του πίνακα είναι συμπληρωμένα, με βάση αυτόν τον αριθμό ανάβει ο ανάλογος αριθμός LEDs (Table~\ref{table:progress_bar}).
Για την υλοποίηση της περιοδικής ανανέωσης των LEDs χρησιμοποιήθηκε ο Timer 0 σε λειτουργία output compare θέτωντας το καταχωρητή OCR0 στη τιμή 150 κατά την αρχικοποίηση του Timer 0. Ο prescaler τέθηκε στην τιμή 256 θέτωντας στο καταχωρητή TCCR0 το bit CS02 ίσο με 1. Η τιμή του OCR0 και του prescaler ενδέχεται να αλλάξουν καθώς δεν έχει ελεγχθει πάνω στη πλακέτα ο ρυθμός ανανέωσης των LEDs και η επίδραση στην απόδοση της επίλυσης του Sudoku.

\begin{table}[h!]
\centering
\begin{tabular}[c]{| c | c |}
\hline
LEDs & Clues found \\
\hline
LEDs off & $< 10$ \\
\hline
LED00 &  $\geq 10$ \\
\hline
LED00-01 & $\geq 20$ \\
\hline
LED00-02 & $\geq 30$ \\
\hline
LED00-03 & $\geq 40$ \\
\hline
LED00-04 & $\geq 50$ \\
\hline
LED00-05 & $\geq 60$ \\
\hline
LED00-06 & $\geq 70$ \\
\hline
LED00-07 & $\geq 80$ \\
\hline
\end{tabular}
% \renewcommand{\arraystretch}{1.2}
\caption{Progress Bar}
\label{table:progressBar}
\end{table}



% \subsection{Αλλαγές σε σχέση με το Εργαστήριο 3}


% \begin{figure}[h!]
% \centering
% \includegraphics[scale=0.75]{lab5_flo.png}
% \caption{Ροοδιάγραμμα του προγράμματος.}
% \end{figure}


\section{Έλεγχος σωστής λειτουργίας}



\section{Συμπεράσματα}


\end{document}